
\documentclass[]{article}
\usepackage{float}
\usepackage{graphicx}
\usepackage[]{subcaption}
\captionsetup[figure]{labelfont={bf,footnotesize},textfont=footnotesize}
\captionsetup[table]{labelfont={bf,footnotesize},textfont=footnotesize}
\usepackage{xcolor}
\usepackage{listings}
\lstset{
  frame=single,
  breaklines=true,
  postbreak=\raisebox{0ex}[0ex][0ex]{\ensuremath{\color{red}\hookrightarrow\space}}
}
\usepackage{hyperref}
\usepackage[]{amsmath}

\newcommand{\TwoRowCell}[2][c]{%
\begin{tabular}[#1]{@{}c@{}}#2\end{tabular}}

\begin{document}

\section{Notes}
Laboratory exercise and data analysis : Programming and statistics

Should introduce the course, perhaps base it more on course plan.
Course: Nuclear Physics and Reactors?
It is a mandatory course for the science faculty students in the last semester of their second year.
The course covers theoretical models that describe atomic nuclei.
The student is also introduced to the detection of nuclear radiation and the experimental part of the field \ref{web:course}.

I was supervising a laboratory exercise in the course, called $\gamma$-spectroscopy.
This is a full day lab which can be split in two parts, before lunch walking through theory of $\gamma$ rays and radiation detectors and after lunch performing measurements and discussing acquired energy spectra.
In the second part different tasks that are to be included in the report is carefully described by the supervisor.
The majority of the tasks cover what is denoted data analysis, i.e. reading in experimental data, calculate specific properties and perform a statistical analysis on the obtained results.
To successfully complete the data analysis the students need to independently obtain knowledge and program a code.
Already during the lab session there was some quarelling from the students.
They want very detailed information of how each calculation should be performed.
Some students point out that they do not know how to program and/or learned sufficient statistics.
In the process of completing the report several students turned to me for detailed descriptions of what code to implement, my feeling was that they were just looking for the answer right away.
In the continuation two parts of the course plan were not really conveyed or \dots. These are:
\begin{itemize}
  \item Planera, genomföra och redovisa experiment.
  \item Värdera experimentella resultat.
  \item Självständigt kunna inhämta nya kunskaper och redovisa dessa i muntligt och
    skriftligt.
\end{itemize}


Students unclear of what/how to do it? Student wants to be given the code directly\dots

Course home page: \url{http://www.fysik.lu.se/english/education/courses/basic-level/fysc01-course-package/nuclear-physics/}

Is the course plan walked through?

Need to expand this part \dots


\subsection{Possible solutions}
Define the aims/objectives of the lab more clearly. Clarify formulas even more.

There is a swift change in the method of learning.
Studying theory (students do this in a certain way) and then they need to acquire the craft/skill of performing data analysis, i.e. daq and statistical analysis.
There is a resistance from the students here.
Rephrase this as a challenge.

Solutions:
\begin{itemize}
  \item Need better motivation to facilitate change
  \item Clarify early on: $\gamma$-spectroscopy and data analysis
  \item Provide some background theory to the data analysis as well (on intro-meeting?)
  \item Strive to center the data analysis in general for the complete course.
  \item In beta lab the intenden learning outcomes are presented in the lab manual.
\end{itemize}




\end{document}
