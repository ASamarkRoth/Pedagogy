
\documentclass[]{article}
\usepackage{float}
\usepackage{graphicx}
\usepackage[]{subcaption}
\captionsetup[figure]{labelfont={bf,footnotesize},textfont=footnotesize}
\captionsetup[table]{labelfont={bf,footnotesize},textfont=footnotesize}
\usepackage{xcolor}
\usepackage{listings}
\lstset{
  frame=single,
  breaklines=true,
  postbreak=\raisebox{0ex}[0ex][0ex]{\ensuremath{\color{red}\hookrightarrow\space}}
}
\usepackage{hyperref}
\usepackage[]{amsmath}

\newcommand{\TwoRowCell}[2][c]{%
\begin{tabular}[#1]{@{}c@{}}#2\end{tabular}}

\begin{document}

\section{Notes}
Laboratory exercise and data analysis : Programming and statistics

Students unclear of what/how to do it? Student wants to be given the code directly\dots

Need to expand this part \dots


\subsection{Possible solutions}
Define the aims/objectives of the lab more clearly. Clarify formulas even more.

There is a swift change in the method of learning.
Studying theory (students do this in a certain way) and then they need to acquire the craft/skill of performing data analysis, i.e. daq and statistical analysis.
There is a resistance from the students here.
Rephrase this as a challenge.

Solutions:
\begin{itemize}
  \item Need better motivation to facilitate change
  \item Clarify early on: $\gamma$-spectroscopy and data analysis
  \item Provide some background theory to the data analysis as well (on intro-meeting?)
  \item Strive to center the data analysis in general for the complete course.
\end{itemize}




\end{document}
